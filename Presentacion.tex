\documentclass[11pt, xcolor={table,xcdraw}]{beamer}
\usetheme[
%%% options passed to the outer theme
    progressstyle=fixedCircCnt,   %either fixedCircCnt, movCircCnt, or corner
    rotationcw,          % change the rotation direction from counter-clockwise to clockwise
    shownavsym          % show the navigation symbols
  ]{AAUsimple}

\definecolor{darkblue}{RGB}{51,51,179}

% If you want to change the colors of the various elements in the theme, edit and uncomment the following lines
% Change the bar and sidebar colors:
\setbeamercolor{AAUsimple}{fg=gray!50 ,bg=darkblue}
\setbeamercolor{sidebar}{bg=gray!20}
\setbeamercolor{frametitle}{fg=darkblue!5,bg=darkblue}
% Change the color of the structural elements:
\setbeamercolor{structure}{fg=darkblue}
% Change the frame title text color:
%\setbeamercolor{frametitle}{fg=darkblue!20}
% Change the normal text color background:
%\setbeamercolor{normal text}{fg=black,bg=gray!10}
% ... and you can of course change a lot more - see the beamer user manual.

\definecolor{mygreen}{rgb}{0,0.6,0}
\definecolor{mygray}{rgb}{0.5,0.5,0.5}
\definecolor{mymauve}{rgb}{0.58,0,0.82}

\usepackage[utf8]{inputenc}
\usepackage[spanish]{babel}
\usepackage[T1]{fontenc}
% Or whatever. Note that the encoding and the font should match. If T1
% does not look nice, try deleting the line with the fontenc.
\usepackage{helvet}

% colored hyperlinks
\newcommand{\chref}[2]{%
  \href{#1}{{\usebeamercolor[bg]{AAUsimple}#2}}%
}

\title{Sistema de control para estación autónoma marítima de monitoreo de ruido ambiente}

\subtitle{Presentación del Trabajo Final de Maestría}  % could also be a conference name

\date{\today}

\author{Esp. Ing. Patricio Bos }

\institute[
%  {\includegraphics[scale=0.2]{aau_segl}}\\ %insert a company, department or university logo
  Dept.\ de electrónica\\
  Facultad de Ingeniería\\
  Universidad de Buenos Aires
] % optional - is placed in the bottom of the sidebar on every slide
{% is placed on the bottom of the title page
  Maestría en Sistemas Embebidos\\
  Facultad de Ingeniería\\
  Universidad de Buenos Aires
  
  %there must be an empty line above this line - otherwise some unwanted space is added between the university and the country (I do not know why;( )
}

% specify a logo on the titlepage (you can specify additional logos an include them in 
% institute command below
\pgfdeclareimage[height=1.5cm]{titlepagelogo}{imagenes/logo_facu_circle} % placed on the title page
%\pgfdeclareimage[height=1.5cm]{titlepagelogo2}{AAUgraphics/aau_logo_new} % placed on the title page
\titlegraphic{% is placed on the bottom of the title page
  \pgfuseimage{titlepagelogo}
%  \hspace{1cm}\pgfuseimage{titlepagelogo2}
}

\definecolor{darkblue}{RGB}{51,51,179}
\setbeamercolor{bgcolor}{fg=white,bg=darkblue}

\setbeamertemplate{navigation symbols}{}
\beamerdefaultoverlayspecification{<+->}

\AtBeginSection[]
{
 \begin{frame}<beamer>
 \frametitle{\textbf{\LARGE{Agenda}}}
 \fontsize{18pt}{18}\selectfont
 \tableofcontents[currentsection]
 \end{frame}
}

\usepackage{setspace}
\usepackage{color}
\usepackage{listings}

\lstset{ %
  backgroundcolor=\color{white},   % choose the background color; you must add \usepackage{color} or \usepackage{xcolor}
  basicstyle=\large,        % the size of the fonts that are used for the code
  breakatwhitespace=false,         % sets if automatic breaks should only happen at whitespace
  breaklines=true,                 % sets automatic line breaking
  captionpos=b,                    % sets the caption-position to bottom
  commentstyle=\color{mygreen},    % comment style
  deletekeywords={...},            % if you want to delete keywords from the given language
  %escapeinside={\%*}{*)},          % if you want to add LaTeX within your code
  %extendedchars=true,              % lets you use non-ASCII characters; for 8-bits encodings only, does not work with UTF-8
  %frame=single,	                   % adds a frame around the code
  keepspaces=true,                 % keeps spaces in text, useful for keeping indentation of code (possibly needs columns=flexible)
  keywordstyle=\color{blue},       % keyword style
  language=[ANSI]C,					% the language of the code
  %otherkeywords={*,...},           % if you want to add more keywords to the set
  numbers=none,                    % where to put the line-numbers; possible values are (none, left, right)
  numbersep=5pt,                   % how far the line-numbers are from the code
  numberstyle=\tiny\color{mygray}, % the style that is used for the line-numbers
  rulecolor=\color{black},         % if not set, the frame-color may be changed on line-breaks within not-black text (e.g. comments (green here))
  showspaces=false,                % show spaces everywhere adding particular underscores; it overrides 'showstringspaces'
  showstringspaces=false,          % underline spaces within strings only
  showtabs=false,                  % show tabs within strings adding particular underscores
  stepnumber=1,                    % the step between two line-numbers. If it's 1, each line will be numbered
  stringstyle=\color{green},     % string literal style
  tabsize=2,	                   % sets default tabsize to 2 spaces
  title=\lstname,                   % show the filename of files included with \lstinputlisting; also try caption instead of title
  morecomment=[s]{/*}{*/}%
}

\usepackage{booktabs}
%\usepackage[table,xcdraw]{xcolor}
\usepackage{tikz}
\def\checkmark{\tikz\fill[scale=0.4](0,.35) -- (.25,0) -- (1,.7) -- (.25,.15) -- cycle;}
\usepackage{multirow}

\begin{document}

%==================================================================================
%  PORTADA
%==================================================================================
\begin{frame}[plain,noframenumbering]
	\begin{center}
	\vspace{5px}	
	\Large\textbf{Maestría en Sistemas Embebidos}\\
	\vspace{5px}
	\large\textbf{Universidad de Buenos Aires}\\
	\vspace{10px}
  \begin{beamercolorbox}[center,sep=1.125ex,dp=1.125ex,ht=18ex, wd=\paperwidth]{bgcolor}
	  \huge\textbf{Sistema de control para estación autónoma marítima de monitoreo de ruido ambiente}\\
    	\vspace{5px}
	  \Large\textbf{Esp. Ing. Patricio Bos}\\
  \end{beamercolorbox}
%	\vspace{10px}
	\vfill
	\vspace{15px}
	\begin{minipage}[t]{0.47\textwidth}
		\begin{flushleft} \large
			\textbf{Director:}\\
			Dr. Ing Ariel Lutenberg
		\end{flushleft}
	\end{minipage}
	\hfill
	\begin{minipage}[t]{0.47\textwidth}
		\begin{flushright} \large
			\textbf{Jurados:} \\
			Dr. Ing. Pablo Gómez \\
			Ing. Juan Manuel Cruz\\
			Mg. Lic Igor Prario\\
		\end{flushright}
	\end{minipage}
%	\vfill
%	\begin{figure}[H]
%		\includegraphics[width=2cm]{./imagenes/logo_facu_circle}
%	\end{figure}	
%	\vspace{5px}
	\end{center}
\end{frame}

%==================================================================================
%  TOC
%==================================================================================
\begin{frame}{\textbf{\LARGE{Agenda}}}
\fontsize{18pt}{18}\selectfont
\tableofcontents
\end{frame}


%==================================================================================
%  MOTIVACIÓN
%==================================================================================
\section{Motivación}

\begin{frame}{\textbf{\LARGE{Motivación}}}
\fontsize{18pt}{18}\selectfont
	\vspace{-.7cm}
	\centering
	\begin{itemize}
	\item ¿Por qué acústica submarina?
	\vspace{15px}
	\item ¿Qué es el nivel de ruido?
	\vspace{15px}
	\item ¿Por qué interesa medirlo?
	\vspace{15px}	
	\item ¿Qué disciplinas lo necesitan?
%	\item 
	\end{itemize}
\end{frame}

\begin{frame}{\textbf{\LARGE{Antecendentes}}}
	\vspace{-.6cm}
		\includegraphics[width=.9\textwidth]{./imagenes/antecedentes.jpg}
\end{frame}

\begin{frame}[c]{\textbf{\LARGE{Objetivo}}}{General}
  \fontsize{18pt}{18}\selectfont
	\begin{itemize}
		\item Prototipo de estación autónoma.
		\vspace{15px}
		\item Medición de señales acústicas.
		\vspace{15px}
		\item Medición de parámetros ambientales.
		\vspace{15px}	
		\item Almacenamiento de la información.
		\vspace{15px}	
		\item Transmisión en tiempo real.
	\end{itemize}
\end{frame}

\begin{frame}{\textbf{\LARGE{Objetivo}}}{Particular}
  \fontsize{18pt}{18}\selectfont
	\vspace{-.7cm}
	\centering
	\begin{itemize}
		\item Desarrollar un firmware de control para la CIAA-NXP.
		\vspace{15px}
		\item Arquitectura multicore modular y flexible.
		\vspace{15px}
		\item Mecanismos de comunicación, sincronización y concurrencia.
		\vspace{15px}	
		\item Interfaz de usuario
		%\vspace{15px}	
		%\item 
	\end{itemize}
\end{frame}
%==================================================================================
%  PLANIFICACIÓN
%==================================================================================
\section{Planificación}

\begin{frame}{\textbf{\LARGE{Diagrama en bloques}}}
%	\vspace{-.6cm}
	\includegraphics<1>[width=\textwidth]{./imagenes/Diagrama_en_Bloques.pdf}
	\includegraphics<2>[width=\textwidth]{./imagenes/Diagrama_en_Bloques_recorte.pdf}
\end{frame}

\begin{frame}{\textbf{\LARGE{Planificación en etapas}}}
	\vspace{-.7cm}
	\begin{figure}[H]
		{\includegraphics[width=\textwidth]{./imagenes/planificacion.png}}
	\end{figure}	
\end{frame}

\begin{frame}{\textbf{\LARGE{Desglose de tareas en AoN}}}
	\vspace{-.7cm}
	\begin{figure}[H]
		{\includegraphics[height=.8\textheight]{./imagenes/AoN.pdf}}
	\end{figure}	
\end{frame}

%==================================================================================
%   METODOLOGÍA
%==================================================================================
\section{Metodología}

\begin{frame}{\textbf{\LARGE{Modelo de ramas}}}{A successfull git branching model}
	\fontsize{16pt}{16}\selectfont
	\vspace{-.9cm}
	\begin{columns}
	  \column{.5\textwidth}
%	  	\vspace{-.9cm}
	  \begin{figure}[H]
    		\includegraphics<1->[height=.8\textheight]{./imagenes/Git-branching-model.pdf}
	  \end{figure}	
	\hfill
	\column{.5\textwidth} 	
	Ramas creadas:
	\vspace{5px}
	  \begin{itemize}[]
		  \item Master
		  \item Develop
		  \item Adquisición
	  	  \item Almacenamiento
	  	  \item Interfaz de usuario
	 	  \item Control
	 	  \item Testing
	  \end{itemize}
	\end{columns}
\end{frame}


\begin{frame}{\textbf{\LARGE{Inter Process Communications}}}
	\vspace{-.7cm}
	\begin{figure}[H]
		\includegraphics[height=.8\textheight]{./imagenes/IPC.png}
	\end{figure}	
\end{frame}

\begin{frame}{\textbf{\LARGE{Protothreads}}}{Adam Dunkel}
	\vspace{-.3cm}
	\fontsize{18pt}{18}\selectfont
	\vspace{-.7cm}
	\centering
	\begin{itemize}
		\item Multitasking cooperativo.
		\vspace{15px}	
		\item Conjunto de macros o co-rutinas.
		\vspace{15px}
		\item Bloqueo sin cambio de contexto.
		\vspace{15px}
		\item Control de flujo más lineal
		\vspace{15px}
		\item 2 bytes de overhead por thread		
	%	\item 
	\end{itemize}
\end{frame}
%==================================================================================
%  IMPLEMENTACIÓN
%==================================================================================
\section{Implementación}

\begin{frame}{\textbf{\LARGE{Modelo de capas}}}
	\vspace{-.7cm}
	\begin{figure}[H]
		\includegraphics[height=.8\textheight]{./imagenes/capas.pdf}
	\end{figure}	
\end{frame}

\begin{frame}[fragile]{\textbf{\LARGE{Arquitectura de módulos}}}{}
	\hspace{-1cm}
  \begin{minipage}[t]{.55\textwidth}
		\begin{lstlisting}[basicstyle=\footnotesize]
		typedef struct {
		
		   cpuID_t cpuID;
		   
		   moduleID_t moduleID;
		   
		   funcPtr_t eventHandler;
		   
		   tick_t period;
		   
		   moduleStatus_t status;
		   
		} module_t;
		\end{lstlisting}
  \end{minipage}\hfill
  \begin{minipage}[t]{.5\textwidth}
		\begin{lstlisting}[basicstyle=\footnotesize]
			typedef struct {
			
			struct {
				cpuID_t cpuID;
		    moduleID_t moduleID;
			} id;

			signal_t signal;

			(void *) data;

		} ipcex_msg_t;
		\end{lstlisting} 
  \end{minipage}
\end{frame}

\begin{frame}{\textbf{\LARGE{Máquina de Estados Finitos}}}{Módulo genérico}
	\vspace{-1.1cm}
	\begin{figure}[H]
		\includegraphics[height=.8\textheight]{./imagenes/MEF_generica.pdf}
	\end{figure}	
\end{frame}





\begin{frame}[fragile]{\textbf{\LARGE{Arquitectura de Firmware}}}{Disparada por eventos}
	%\vspace{-.7cm}
	%\begin{columns}[t]
	 % \column{.7\textwidth}
	  \begin{lstlisting}[basicstyle=\footnotesize]
int main(void)
{
   bool_t goToSleep = FALSE;

   prvSetupHardware();
   FR_register_all_modules();
   FR_broadcast_signal(sig_init);

   while(TRUE) {  // the main loop

      goToSleep = FR_dispatch_tasks();

      if (goToSleep == TRUE)
         __WFI();
   }
   return 0;
}
\end{lstlisting}
	  %\column{.45\textwidth}
	 % otra
	%\end{columns}
\end{frame}




\begin{frame}{\textbf{\LARGE{Interfaz de usuario}}}{Pantalla principal}
	\vspace{-.7cm}
	\centering
	\begin{figure}[H]
		\includegraphics[width=\textwidth]{./imagenes/interfaz_detalles.pdf}
	\end{figure}	
\end{frame}


\begin{frame}{\textbf{\LARGE{Interfaz de usuario}}}{Modo configuración}
	\vspace{-.7cm}
	\centering
	\begin{figure}[H]
		\includegraphics[width=\textwidth]{./imagenes/interfaz_config_detalle.png}
	\end{figure}	
\end{frame}


%==================================================================================
%  TESTING
%==================================================================================
\section{Testing}

\begin{frame}{\textbf{\LARGE{Testing}}}{Niveles de abstracción}
  \fontsize{18pt}{18}\selectfont
%	\vspace{-.7cm}
	\centering
	\begin{itemize}
		\item Pruebas unitarias.
		\vspace{20px}
		\item Pruebas funcionales.
		\vspace{20px}
		\item Pruebas de sistema.
%		\vspace{15px}	
%		\item Interfaz de usuario
		%\vspace{15px}	
		%\item 
	\end{itemize}
\end{frame}

\begin{frame}{\textbf{\LARGE{Testing}}}{Pruebas unitarias}
	\vspace{-.7cm}
	\begin{figure}[H]
		\includegraphics<1>[width=1\textwidth]{./imagenes/TestUnitario.png}
		%\includegraphics<1>[height=1\textheight]{./imagenes/TestUnitario.png}
	\end{figure}	
\end{frame}

\begin{frame}{\textbf{\LARGE{Testing}}}{Pruebas de sistema: plantilla}
	\vspace{-.7cm}
	\begin{figure}[H]
		\includegraphics[width=1\textwidth]{./imagenes/UseCaseTemplate.pdf}
	\end{figure}	
\end{frame}

\begin{frame}{\textbf{\LARGE{Testing}}}{Pruebas de sistema: casos de uso}
	\vspace{-.75cm}
\begin{columns}
  \column{.53\paperwidth}
  \hspace{.5cm}
	\begin{figure}[H]
	  \includegraphics[height=.8\textheight]{./imagenes/UseCase.png}
	\end{figure}	
	\hfill
	\column{.47\paperwidth} 	
	Casos de uso:
	\vspace{5px}
	  \begin{itemize}[]
		  \item UC01: Adquisición autónoma
		  \item UC02: Cambio de período
		  \item UC03: Cambio de perfil
		\end{itemize}
	\end{columns}
\end{frame}

\begin{frame}[t]{\textbf{\LARGE{Testing}}}{Caso de uso UC01}
	\vspace{-.7cm}
	\begin{figure}[H]
	  \includegraphics[width=.77\textwidth]{./imagenes/UseCase_detalle1.png}
	\end{figure}	
\end{frame}

\begin{frame}[t]{\textbf{\LARGE{Testing}}}{Caso de uso UC02}
	\vspace{-.7cm}
	\begin{figure}[H]
	  \includegraphics[width=.77\textwidth]{./imagenes/UseCase_detalle2.png}
	\end{figure}	
\end{frame}

\begin{frame}[t]{\textbf{\LARGE{Testing}}}{Caso de uso UC03}
	\vspace{-.7cm}
	\begin{figure}[H]
	  \includegraphics[width=.77\textwidth]{./imagenes/UseCase_detalle3.png}
	\end{figure}	
\end{frame}



%==================================================================================
%  DEMOSTRACIÓN
%==================================================================================
\section{Demo}

%\begin{frame}{\textbf{\LARGE{Modelo de ramas}}}
%	\vspace{-.7cm}
%	\begin{figure}[H]
%		{\includegraphics[height=.8\textheight]{./imagenes/Git-branching-model.pdf}}
%	\end{figure}	
%\end{frame}

%==================================================================================
%  CONCLUSIONES
%==================================================================================
\section{Conclusiones}

\begin{frame}{\textbf{\LARGE{Resultados obtenidos}}}
\fontsize{18pt}{18}\selectfont
	\vspace{-.7cm}
	\centering
	\begin{itemize}
  \item Metodología de trabajo.
	\vspace{15px}
	\item Arquitectura modular multicore.
	\vspace{15px}
	\item Mecanismos de control y despacho de tareas y eventos.
	\vspace{15px}
	\item 4 módulos funcionales.
	\vspace{15px}	
	\item Documentación completa.
%	\item 
	\end{itemize}
\end{frame}

\begin{frame}{\textbf{\LARGE{Cumplimiento}}}{Requerimientos funcionales del sistema}
%\fontsize{18pt}{18}\selectfont
\vspace{-.8cm}
\begin{table}[!htpb]
\resizebox{\textwidth}{!}{%
\centering
\begin{tabular}{@{}lc@{}}
\rowcolor[HTML]{9AFF99} 
\begin{tabular}[c]{@{}l@{}}2.1 El sistema debe adquirir datos de un array de sensores de \\ temperatura a intervalos regulares con un período de adquisición \\ seleccionable.\end{tabular} &  \checkmark \\
 &  \\
 \rowcolor[HTML]{FFCCC9} 
\begin{tabular}[c]{@{}l@{}}2.2 El sistema debe adquirir datos de un anemómetro a intervalos \\ regulares con un período de adquisición seleccionable.\end{tabular} &  X \\
 &  \\
\rowcolor[HTML]{9AFF99} 
\begin{tabular}[c]{@{}l@{}}2.3 El sistema debe almacenar los datos de temperatura y velocidad\\ de viento adquiridas junto con una marca de tiempo identificatoria \\ en un medio físico no volátil.\end{tabular} & \checkmark \\
 &  \\
\rowcolor[HTML]{9AFF99} 
\begin{tabular}[c]{@{}l@{}}2.4 El sistema debe poder operar con dos perfiles de consumo de \\ energía máximo desempeño y mínimo consumo de energía.\end{tabular} & \checkmark \\
 &  \\
 \rowcolor[HTML]{9AFF99} 
\begin{tabular}[c]{@{}l@{}}2.5 El sistema debe contar con una interfaz serie cableada que \\ permita realizar operaciones de configuración y mantenimiento.\end{tabular} & \checkmark \\
 &  \\
\end{tabular}
}
\end{table}
\end{frame}

\begin{frame}{\textbf{\LARGE{Trabajo futuro}}}
\fontsize{18pt}{18}\selectfont
	\vspace{-.7cm}
	\centering
	\begin{itemize}
	\item Completar requerimientos y funcionalidades.
	\vspace{15px}
	\item Analizar distintas configuraciones.
	\vspace{15px}
	\item Servidor de integración continua.
	\vspace{15px}	
	\item Diseñar hardware \textit{ad-hoc}.
	\vspace{15px}	
	\item Completar objetivos generales.
%	\item 
	\end{itemize}
\end{frame}


%==================================================================================
%  PREGUNTAS
%==================================================================================
\begin{frame}[plain,c]
%\frametitle{A first slide}
\begin{center}
\usebeamerfont*{frametitle} %\usebeamercolor[fg]{frametitle}
\Huge ¿Preguntas?
\end{center}

\end{frame}


%==================================================================================
%  APÉNDICE
%==================================================================================
\begin{frame}[fragile]{\textbf{\LARGE{Protothreads}}}{Multitasking cooperativo}
	\vspace{-.7cm}
	\begin{verbatim}
		struct pt { unsigned short lc; };
		
		#define PT_THREAD(name_args)  char name_args

		#define PT_BEGIN(pt)          switch(pt->lc) { case 0:

		#define PT_WAIT_UNTIL(pt, c)  pt->lc = __LINE__; \
		                              case __LINE__: \
		                              if(!(c)) return 0

		#define PT_END(pt)            } pt->lc = 0; return 2

		#define PT_INIT(pt)           pt->lc = 0
	\end{verbatim}
\end{frame}

\begin{frame}[fragile]{\textbf{\LARGE{Protothreads}}}{Multitasking cooperativo}
	\vspace{-.6cm}
  \begin{minipage}{.5\textwidth}
    \begin{lstlisting}[frame=tlrb,basicstyle=\footnotesize,label={lst:proto1}]{Protothreads}
static
PT_THREAD(example(struct pt *pt))
{
  PT_BEGIN(pt);
  
  while(1) {
    PT_WAIT_UNTIL(pt,
      counter == 1000);
    printf("Threshold reached\n");
    counter = 0;
  }
  
  PT_END(pt);
}
    \end{lstlisting}
  \end{minipage}\hfill
  \begin{minipage}{.5\textwidth}
    \begin{lstlisting}[frame=tlrb,basicstyle=\footnotesize,label={lst:proto2}]{Traducción}
static
char example(struct pt *pt)
{
  
  switch(pt->lc) { case 0:
 
  while(1) {
    pt->lc = 12; case 12:
    if(!(counter == 1000)) return 0;
    printf("Threshold reached\n");
    counter = 0;
  }
  } pt->lc = 0; return 2;
}
    \end{lstlisting}
  \end{minipage}
\end{frame}



\begin{frame}{\textbf{\LARGE{Máquina de Estados Finitos}}}{Módulo adquisición}
	\vspace{-1.1cm}
	\begin{figure}[H]
		\includegraphics[height=.8\textheight]{./imagenes/MEF_adquisicion.pdf}
	\end{figure}	
\end{frame}

\begin{frame}{\textbf{\LARGE{Máquina de Estados Finitos}}}{Módulo HMI}
	\vspace{-1.1cm}
	\begin{figure}[H]
		\includegraphics[height=.8\textheight]{./imagenes/MEF_HMI_2.pdf}
	\end{figure}	
\end{frame}

\begin{frame}{\textbf{\LARGE{Máquina de Estados Finitos}}}{Módulo almacenamiento}
	\vspace{-1.1cm}
	\begin{figure}[H]
		\includegraphics[height=.8\textheight]{./imagenes/MEF_sdCard_2.pdf}
	\end{figure}	
\end{frame}

\end{document}
